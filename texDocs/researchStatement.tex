\documentclass{article}
\usepackage{graphicx} % Required for inserting images
\usepackage{geometry}

\title{Research Statement}
\author{Jesse Wheeler}
\date{\today}

\begin{document}

\rule{0mm}{1mm}
\vspace{-20mm}

%\hfill{\small \today}

\vspace{1mm}

%\begin{center}
%\hfill
\rule{0mm}{1mm}\hspace{5cm}{\Large {\bf Research Statement}}

\vspace{3mm}
\rule{0mm}{1mm}\hspace{6.1cm}{\large {\bf Jesse Wheeler}}

\vspace{4mm}

% \noindent Dynamic systems are the subject of many ecological and financial studies.
\noindent Recent advancements in data collection and storage have led to an abundance of complex datasets.
Often, these datasets include time-dependent measurements, resulting in interdependent observations that render many traditional statistical modeling approaches ineffective.
This issue is particularly prevalent in nonlinear dynamic systems that are frequently of scientific or industrial interest.

A modern approach to modeling time-dependent observations is the use of state-space (or mechanistic) models \cite{durbin12}.
These models require the proposal of mathematical equations that describe how unobservable states of the system evolve over time and how measurements of the system variables are obtained.
State-space models are appealing because they allow researchers to incorporate mechanisms that reflect our current scientific understanding of the system in question.
This capability enables researchers to estimate the effects of potential interventions on the system, provides a framework for statistical testing of our current understanding against alternative explanations, and facilitates inference on unobservable variables that may be of scientific interest.

Despite the utility of these models, statistical inference of a dynamic system using state-space models remains a challenging task \cite{auger16}.
The primary goal of my current research is to expand the current state-of-the-art for state-space modeling.
In particular, I develop methodology, theory, and software for likelihood based inference of state-space models with the {\it plug-and-play} property, which is that one only needs the ability to simulate from the model in order to perform inference \cite{breto09}.
Algorithms with the plug-and-play property enable researchers to fit models that align with their scientific objectives instead of being limited to those that are statistically convenient \cite{wheeler24}.

Applying state-space models to data requires close collaboration with experts from pertinent scientific disciplines.
For example, conducting a statistical analysis of an infectious disease outbreak relies heavily on expert opinion regarding the specific disease \cite{wheeler24}.
A pertinent example from my ongoing projects is the investigation of population dynamics in freshwater Daphnia species.
In this endeavor, I closely collaborate with colleagues in the Ecology department, leveraging their knowledge of Daphnia.
This collaboration has honed my skills as an effective statistical consultant in ecological research projects, a competency I look forward to further refine through future research collaborations.

% An important outcome of this line of research is
% State-space models are particularly useful in studying population Ecology.

\section{State-space modeling in higher dimensions}\label{sec:methods}

The nonlinear, stochastic nature of many dynamic systems of scientific interest make the state-space modeling approach challenging.
Contemporary approaches that have been successfully used for nonlinear, low-dimensional systems are based on sequential Monte Carlo (SMC) techniques \cite{ionides06,andrieu10,ionides15}.
As the dimensionality of the system increases, however, the approximation error of SMC increases exponentially \cite{bengtsson08,snyder08}.
This necessitates the development of algorithms that can be used to perform inference on high-dimensional nonlinear state-space models.
Recent progress in this area has been made---including my own research projects \cite{ionides22,wheeler24}---yet several open problems remain.
Furthermore, existing plug-and-play algorithms used to model these systems are computationally expensive, making any improvements in computational speed a desirable outcome.
% Recent research by my research group have enabled some progress in this area \cite{ionides22}.

A simplifying assumption that can often be made for high-dimensional, nonlinear dynamic systems is approximate independence between measurement units.
Data collected from these systems are sometimes called panel or longitudinal data.
Each time-series in the panel may be relatively short, so that inference on an underlying model must combine information across the units.
Differences between units may be of direct inferential interest or may be a nuisance for studying the commonalities.
One example includes fitting parameters that are both shared across units in the panel and also those that are specific to each unit.
Current state-of-the-art plug-and-play algorithms available for fitting models of this type of model treat panel state-space models similar to their lower-dimensional equivalents \cite{breto20}, and can therefore still suffer from particle depletion \cite{snyder08}, resulting in the need for expensive numerical calculations.

One of my current research projects aims to reduce the computational burden associated with fitting state-space models to panel data.
An algorithm I have designed, called the block panel iterated filter (BPIF), does this by accounting for the independence between the likelihood of a given unit in a panel and the unit-specific parameters of other units.
While the theory for this algorithm is still being developed, it has demonstrated to be an improvement over existing approaches on simulated data and on examples that have been attempted by undergraduate students I have mentored \cite{yang23,sun24}.

A weakness of maximum likelihood estimation is that estimated parameters may not make sense in the context of the model and data in question \cite{lecam90}.
For mechanistic models, this is particularly troublesome because estimated parameters should provide a quantitative description of the data while retaining a meaningful mechanistic interpretation \cite{wheeler24}.
In high dimensional systems, the chance of finding parameters that maximize the likelihood in implausible regions of the parameter space increases \cite{li24}.
For this reason I plan on extending existing iterated filtering algorithms \cite{ionides15,ionides22} to enable the maximization of penalized likelihood \cite{cole13}, which helps address issues of parameter instability that arise in likelihood maximization.
Additionally, penalized likelihood may enable the estimation of random effects in high-dimensional state-space models, an open research question.

% Finally, recent advancements in automatically differentiable particle filters (ADPF) by a student that I mentored as an undergraduate student and my research advisory shows promise as an computational improvement over existing methods on low-dimensional systems.
% \section{State-space modeling of infectious diseases}\label{sec:epi}

% The ongoing global COVID-19 pandemic has sparked interest in the statistical modeling of infectious diseases.
% State-space models are well-suited for epidemiological systems as researchers have long used partially observed dynamical models to describe the development of an infectious disease outbreak.
% The methodological, theoretical, and computing developments mentioned in Sections \ref{sec:methods} and \ref{sec:software} provide epidemiologists with a new set of tools for modeling infectious disease outbreaks.
% In \cite{wheeler24}, I provide practical advice on how to use these tools and avoid common mistakes that are made in infectious disease modeling.

\section{Additional Research}

State-space models have additional uses beyond the class of mechanistic models that are popular in Ecology.
Likelihood maximization for auto regressive moving average (ARMA) models, for instance, is traditionally done by reparameterizing the model into an equivalent linear Gaussian state-space form \cite{gardner80, durbin12}.
% In this case, the state-space model is a convenient approach to fitting a model rather than being of direct interest itself.
Despite the wide use of ARMA models across various scientific disciplines, current algorithms for maximum likelihood estimation of ARMA models have weaknesses that are not well known.
In one of my working papers, I demonstrate how existing optimization strategies for ARMA models often result in parameter estimates corresponding to local---rather than global---maximum of the likelihood surface.
To rectify this, I propose a random restart algorithm that frequently results in higher likelihoods than standard alternatives \cite{wheelerARMA}.

While my current research is focused on state-space modeling of dynamic systems, I am interested in many other statistical topics and am open to new research topics and collaborations.
Some themes common to each of my research interests are transparency and reproducibility.
I believe that many of the challenges faced by various scientific disciplines are a direct result of the poor practice of these principles.
I also believe that statisticians must play a more active role in promoting and executing these principles.
As such, one of my primary research goals is to positively impact the scientific community by writing papers and developing software that encourage and enable other researchers to incorporate these principles in their own work.

\section{Software}\label{sec:software}

The development of theory and methods for likelihood based inference of state-space models requires development of new software.
This section briefly describes some of the open-source software packages to which I have contributed.

\begin{itemize}
    \item \texttt{panelPomp}: This package provides a framework for developing high-dimensional state-space models under the assumption of independence between panel units.
    This package is publicly available on both CRAN and GitHub.
    While I was not the original creator of the package, I am currently the primary package maintainer and developer.

    \item \texttt{arima2}: This package provides useful functions for fitting auto-regressive, integrated, moving-average (ARIMA) models in \texttt{R}. The most important function of this package is the \texttt{arima} function, which fits ARIMA models using a multiple restart algorithm that frequently results in models with higher likelihoods than other alternatives \cite{wheelerARMA}.
    I am the creator and primary contributor of this package, which is publicly available on both CRAN and GitHub.

    \item \texttt{pomp} and \texttt{spatPomp}: These are both popular packages that enable inference for mechanistic models. The \texttt{pomp} package in particular is used by thousands of researchers across a variety of disciplines. I am an active contributor to both of these open source projects.

    % \item \texttt{spatPomp}: This package provides a framework for developing spatiotemporal partially observed Markov process (SpatPOMP) models for data analysis. An implementation of the method described in \cite{ionides22} is found in this package.
    % My contributions to this package include the implementation of new package features, as well as some bug fixes.

    % \item \texttt{pomp}: This widely used package provides a framework for developing partially observed Markov process (POMP) models; the \texttt{spatPomp} package described previously is an extension of this package.
    % My contributions to this package are bug fixes.
\end{itemize}

\bibliography{myWorks.bib,otherWorks.bib}
\bibliographystyle{ieeetr}

\end{document}

