\documentclass{article}
\usepackage{graphicx} % Required for inserting images
\usepackage{geometry}
\usepackage[table]{xcolor}
\newcommand\TODO[1]{\textcolor{red}{[TODO: #1]}}


\title{Teaching Statement}
\author{Jesse Wheeler}
\date{\today}

\begin{document}

\rule{0mm}{1mm}
\vspace{-20mm}

%\hfill{\small \today}

\vspace{1mm}

%\begin{center}
%\hfill
\rule{0mm}{1mm}\hspace{5cm}{\Large {\bf Teaching Statement}}

\vspace{3mm}
\rule{0mm}{1mm}\hspace{6.1cm}{\large {\bf Jesse Wheeler}}

\vspace{4mm}

% \noindent Dynamic systems are the subject of many ecological and financial studies.

\noindent Thoughout my academic career I have seen many differing approaches to teaching, from both the perspective of a student and from the perspective of an instructor.
These experiences have all shaped my teaching philosophy in differeing ways, yet all have reaffirmed the importance of the teaching duty of an academic.
Effective instructors have the potential to completely transform a students life for the better, as several professors did for me when I was a student.
As such, developing skills of an effective instructor is a responsibility I take very seriously.

Effective teaching extends far beyond merely presenting raw material to students.
As an educator, I believe that students learn most effectively when they can connect theoretical ideas to real-world problems.
An effective way to do this is to find recent published articles that are related to course topics to use as motivating examples or discussion points.
By presenting historic backgrounds and the motivations behind certain concepts, I can enrich students' learning experiences and help them grasp the broader significance of the subject matter.

Recognizing that each student comes from a unique background and brings diverse experiences to the classroom, I strive to present material in various ways to accommodate different learning styles.
Adaptability in teaching is crucial; what works well for one student may not resonate with another.
Therefore, I employ multiple teaching strategies to ensure effective engagement with the material for all students.
Additionally, I have seen firsthand how adapting future lessons based on frequent assessments—such as homework, quizzes, and tests—can enhance learning outcomes.
For instance, revisiting challenging topics before progressing ensures a solid understanding of the current material.

In my teaching experience, I have observed that well-crafted homework assignments significantly enhance learning outcomes without causing undue stress.
For example, I taught a lab course for the same undergraduate statistics course for two semesters, each time supervised by different faculty members.
One professor assigned long and technically challenging homework assignments each week, presumably with the idea that students would eventually benefit from struggling through difficult problems on their own.
The other instructor, however, typically designed guided homework assignments that presented difficult ideas in a progressive and educational way.
Students that were exposed to the thoughtfully crafted homework assignments not only were less stressed thoughout the semester, but they could also clearly demonstrate a better understanding of course materials at the end of the semester.

Statistics methodology and theory can often appear abstract and challenging, especially for students whose primary discipline is not statistics.
Example-based learning can make abstract concepts more relatable.
For instance, I use specific datasets to illustrate the relevance and application of the topics I teach, demonstrating both the necessity of particular theories or methodologies and their practical use.
Student-led projects further bridge the gap between theory and practice; well-designed projects and assignments encourage students to engage with the material more meaningfully, promoting critical thinking and practical application of course concepts.
Through project-based learning, students can delve deeper into subjects, collaborate with peers, and develop essential skills for their future careers.

The advent of generative AI, such as ChatGPT, presents both opportunities and challenges in teaching statistics.
These tools are transforming how students work and learn, necessitating adaptation by effective instructors.
Rather than ignoring AI, it is important to teach students how to use these tools responsibly and effectively in both research and learning contexts.
For example, in the Stats 531 course I instructed at the University of Michigan, ChatGPT was permitted as long as students could demonstrate their engagement beyond merely copying answers.
They were required to explain their learning process and application of knowledge, encouraging active and thoughtful interaction with the material.
While AI tools can be beneficial, it is crucial for educators to evolve their teaching methods to prevent students from relying solely on AI-generated answers.

In conclusion, my teaching philosophy revolves around creating a dynamic and inclusive learning environment that caters to the diverse needs of students.
Through the integration of real-world examples, adaptive teaching methods, thoughtfully designed assignments, project-based learning, and the judicious use of generative AI, I aim to foster a deep and meaningful understanding of statistics among my students.

\bibliography{myWorks.bib,otherWorks.bib}
\bibliographystyle{ieeetr}

\end{document}

